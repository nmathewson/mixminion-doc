\documentclass{article}
\begin{document}
\title{The MixMinion architecture}
\maketitle

\subsubsection{HISTORY}
\begin{verbatim}
 6 March 2002 (Nick: Re-wrapped to be readable in Unix)
14 March 2002 (George: Reworked to reflect discussion)
\end{verbatim}

\section{Abstract}
\section{Introduction}

[Q Roger is reworking the introduction]

\section{Requirements}

MixMinion is a classic store and forward mix design. A number of high
level requirements have been identified, and guided the its design:

\begin{quote}
     ``The system must be resistant against passive traffic analysis.''
\end{quote}

We define traffic analysis as the following: Given that an adversary
sees a message sent or received he/she tries to respectively find who
has received or sent the message. Resistance to traffic analysis means
that the cost of doing so is impractically large. Passive traffic
analysis involves an adversary that has the power to observe all the
network links. This must be true even if a subset of the mixes is
colluding with the adversary.

\begin{quote}
     ``The system must be resistant against active traffic analysis.''
\end{quote}

In that case an adversary has the ability to delete, inject, or replay 
any message into the system. We should try to make any attempt to
introduce malformed messages into the network fruitless. We should
also try to avoid techniques that could amplify active attacks to
become denial of service attacks.

\begin{quote}
     ``The system must make traffic confirmation expensive.''
\end{quote}

Traffic confirmation involves confirming or rejecting with some degree
of confidence the hypothesis that two parties are communicating with
each other. We should aim for MixMinion to provide at least Plausible 
Deniability. Attackers should not be able to convince themselves or a
court that two people are corresponding with each other ``beyond
reasonable doubt.'' If possible there should be ``safe'' modes that would 
make traffic confirmation as difficult as traffic analysis, at the
expense of latency and bandwidth.

\begin{quote}
     ``The system should be secure against a fraction of corrupt 
     collaborating mix servers.''
\end{quote}

For this reason we aim at making intermediate servers unaware of 
the full route of the message, its destination or the sender. We
should also aim at making the length of the full route and their
particular position undetectable except in particular documented
cases. 

\begin{quote}
     ``The system should provide forward security where possible, be
     designed in such a way that it is possible to build forward
     secure systems on top of it.''
\end{quote}

We use forward secure channels between mixes. 

\begin{quote}
     ``The system should support feature that provide a certain degree
      of unobservability.''
\end{quote}

That can be archived by adding dummy traffic and traffic shaping but
is unlikely to be needed or used by all users. MixMinion should take
into account that some of the traffic's only purpose is to act as
cover traffic and manage it efficiently.

\begin{quote}
     ``The system should be modular, and allow for different types of
     applications to be built on top of it.''
\end{quote}

The basic transport mechanism is common for all applications. The
protocol itself includes as little as possible application level
details. Some basic multiplexing is included in MixMinion to direct
the message to the appropriate application module, since multiplexing
at the TCP layer would be observable reducing the anonymity sets
provided by the system.

\begin{quote}
     ``The design of the system should minimize the potential for
     Denial of Service attacks.''
\end{quote}

The design should be careful to avoid resources being used or reserved
for a long time. The two mechanisms that could affect the performance
of classical mixes are replay prevention and return addresses
state. MixMinion should use techniques that make sure to minimize the
amount of data and the time the data is kept on the mix nodes. 

\begin{quote}
     ``The design should be easy to be adapted to run on crypto
     hardware. This could be done for efficiency or security reasons.''
\end{quote}

The computational bottleneck in a mix is likely to be the asymmetric
cryptographic operations. A hardware co-processor could be useful if
one want to design high-throughput systems. The tamper-resistance
offered by some crypt-processors can also be used to secure the mix
against adversaries that want to get the private keys out. If such a
box is programmed to only decrypt each message once and never reveal
the key it cannot be used to trace past or future traffic.

\begin{quote}
     ``The design and implementation should be friendly toward formal
     specification, formal evaluation or specification.''
\end{quote}

Formal evaluation could be used to make sure that mixes adhere to
particular protection profiles, such as the ones proposed by Kai and ?
[PP for Mixes]. The protocols and formats used by the Mixes should
also be friendly to formal specification and analysis.

\begin{quote}
     ``Only clients that benefit from the system properties should be
     obliged to know how it works, or to have specialized tools to use
     it.''
\end{quote}

This will allow an easier and therefore wider deployment of the
system, and ease of integration with other products.

\begin{quote}
     ``The cryptographic components of the system should be well
     understood and standardized where possible.''
\end{quote}

We use the PKCS\#1 RSA-OAEP, RSA-PSS standard, SHA-1 Standard and
Rijendal in counter mode. 

\subsection{Rejected design decisions}

[Q Feel free to fill the gaps on the rejected design, or add new ones]

\begin{itemize}
\item Logs
\item Extensions
\item Receipts
\item Fancy ElGamal Schemes + verifiable mixing.
\item Connection oriented mixes (Onion routing)
\item Multiple inter-mix communication schemes.
\end{itemize}

\section{Specifications}

The key words "MUST", "MUST NOT", "REQUIRED", "SHALL", "SHALL
NOT", "SHOULD", "SHOULD NOT", "RECOMMENDED",  "MAY", and
"OPTIONAL" in this document are to be interpreted as described in
RFC 2119.\} Need to correct the document to comply with 2117.

\subsection{Overview}

The main areas that need to be defined in order to have an
implementable system are the basic packet formats and how to process
them, the transport protocols and the administrative functions such as
key and capability discovery. 

\subsection{The packet format}

The message format of MixMinion packets is divided in
2 parts. The Main Header (MH), contains information encrypted under 
the public key of the mix it is addressed to. The second part is the 
payload that contains the message to be forwarded to the next hop or 
to the local application. The Payload is encrypted using a symmetric 
encryption method.

The exact format of the Master Header is:

\begin{verbatim}
Master Header:

- Major Version      - 1 byte
- Minor Version      - 1 byte
- Master Secret (MS) - 16 bytes
- Timestamp          - 8 bytes
- Size               - 2 bytes
- Type               - 2 bytes
- Return Level       - 1 byte
- Reuse Level        - 2 bytes
- Address Type       - 2 bytes
- Next Address       - 16 bytes
- Dest. Port         - 2 bytes
- Expected Key       - 20 bytes

Total size: 73 bytes
\end{verbatim}

In the rest of the document we will refer to these fields as:

\begin{verbatim}
Major Version, Minor Version: V [2 bytes]
Master Secret: MS [16 bytes]
Timestamp: T [8 bytes]
Size: S [2 bytes]
Type: Tp [2 bytes]
Return Level: Rt [1 byte]
Reuse Level: Ru [2 bytes]
Address Type, Next Address, Dest. Port, Expected Key: A [20bytes]
\end{verbatim}

We denote a particular header composed of the above elements as:

\begin{verbatim}MH(V, MS, T, S, Tp, Rt, Ru, A)\end{verbatim}

The Master Header MUST be encrypted using the RSAEP-OAEP method with
an RSA key of modulus 1024 bits as described in the PKCS\#1
standard. The size of the encrypted header block is 128 bytes. The
OAEP transform is applied using SHA-1 and the MGF1 Mask 
generation function as described in the PKCS\#1 standard, and the 
parameter P is fixed to [what]. This construction 
makes sure that any modification of the Master Header is detectable by 
the mix, and MUST result in the packet being discarded.

The \emph{Major and Minor Versions} are present in the header to help the mix
server interpret the rest of the packet. If the server does not recognize
the major or the Minor Version it MUST drop the packet (and check
its policy files to see if they are clear about the versions supported).

The \emph{Master Secret} is a shared secret between the user and the
mix. It SHOULD be forgotten as soon as the processing of the packet
has ended. It SHOULD never be used directly for any operation. Instead
it is hashed with particular strings to extract keys or seeds for the
random number generators.

The \emph{timestamp} SHOULD be at a coarse granularity (a day) so that
it cannot be used by corrupt mixes to link messages together. It
SHOULD refer to GMT time worldwide so that a corrupt mix cannot guess
in which time zone the sender is. The value of timestamps should be an
integer representing the number of whole days since Jan 1, 1970, GMT.

Each packet generates a unique signature that is stored on the mix to avoid 
processing the same packet twice. The signature is computed by hashing the
encrypted Master Header of the message (first 128 bytes). It MUST be kept for
as long as the message has not expired.

All the conversions from integer to byte array or the other way MUST
follow the procedures I2OSP and OS2IP described in PKCS\#1. The length
of Packet field SHOULD contain a fixed value of 62kb (62*1024 bytes)
when the packet is transiting through the network. Another value N
(smaller than 62kb) MAY be specified if the message is to be processed
locally on the mix. In that case only the N first bytes of the body
are to be given to the application.  The size of 62kb has been chosen
to fit into the maximum size UDP packet although UDP is not by default
used for transport.  [Is 64kb a bit large for each packet? This is
much more than the average email and even with 20 hops (which should
cover most of the planet anonymity set wise) you still have around
60kb remaining. What about going down to 32kb or 16kb?]

The \emph{type field} of the header allows unobservable multiplexing of
applications using anonymous communications. Some values have special
meaning. Type 0 means that the packet is not to be processed by any
module and MUST be sent to the address specified in the header. For
any other value the body of the packet SHOULD given to a module taking
care of packets of this category. The values 1 to 255 are reserved for
management functions defined by MixMinion standards such as deletion
of cover traffic and packet reconstruction. The values from 256 to
32767 should be registered and well known. Values greater than 32767
can be used freely but could result in clashes.

The \emph{reuse level} allows an intermediary mix to let through the same
headers many times. It is important to observe that multiple use
return blocks offer much weaker anonymity than one use return
blocks. They allow the server to know that the address is reused and
link messages together. Server can know where it is on the path. They
also allow servers to perform message tagging in order to detect the
message later on the path if it goes through another corrupt
server. Multiple use reply blocks should be avoided is the
functionality can be implemented in another way. The Reuse level is
set to zero if the message is not a multiple use return address.

The \emph{return level} field SHOULD be used on the last header of a
return block (usually addressed to the creator of the block) to remind
them how many levels of encryption are to be stripped.

The \emph{Address Type} field MUST be used to specify what address type is
following. It SHOULD be full of zero valued bytes in case the message
is to be processed locally. The usual type of address should be IPv4
(using up 4 bytes) but ``ephemeral address'' schemes MAY also be
supported to provide return addresses with forward security. The
Address field is 16 bytes long to accommodate long addresses used by
IPv6. The Port Number specifies the TCP port that listens for
MixMinion messages. The hash of the signature key of the next hop MAY
also be included in the header. If it does not match the signature
returned by the next address the message MUST be dropped. If the hash
is not specified it should be filled with zero valued bytes, and that
signifies that there SHOULD NOT be a check against the recipients key.

\subsection{Encoding messages}

A client should get hold, using some method, of a list of mixes along with 
their capabilities and public keys. Special protocols should be defined to 
perform these administrative operations, so that they do not leak any 
information about potential routes, therefore compromising anonymity.

The client then chooses a number of mixes $M_0$ to $M_(N-1)$ with keys
$K_0$ to $K_(N-1)$ and makes sure that consecutive pairs of them have
policies that allow to communicate with each other. Assume that the
message $M$ is to be sent to final address $M_0$. $M$ is of length $L$
bytes and MUST have a maximum length of: \( L < 62kb - N*128 + 1 \)

The client also needs to choose the final type of the message to guide
how $M_0$ should treat the message.

\subsubsection{Definitions}

\begin{verbatim}
Define:
- if B is a byte array, B[i:j] (j bytes) is sub array starting at 
  byte i with length j.
- R(n) (n bytes) Generates n random bytes 
- Z(n) (n bytes) Generates n zero bytes
- Len(M) (2 bytes) is the length of message M (* bytes)
- x|y (Len(x)+Len(y) bytes) denotes x concatenated with y.
- PAD(M,L) (L bytes) pads the message M (Len(M) <= L) to length L

- H(M) (20 bytes) is the SHA-1 hash of M (* bytes)

- RSA_OAEP_Encrypt(K,M) (128 bytes): the encryption of M (73
  bytes) using the public key K as described in PKCS\#1.
- RSA_OAEP_Decrypt(K,M) (73 bytes) Gives the decryption of the
  message M (128 bytes) under the private key corresponding to K.
- Encrypt(K,M) (Len(M) bytes) Rijendal in Counter mode encryption 
  of message M using key K. 
- Decrypt(K,M,i,j) (j bytes) Rijendal counter mode decryption 
  using the key material byte i to j. Len(M) = j.
\end{verbatim}

\subsubsection{Encoding Anonymous Messages}

The inputs are a list of Mixes ($M_0$ to $M_{N-1}$) and their public
keys ($K_0$ to $K_{N-1}$), and the final type of the message (TP). The
message to be sent is M.

Encoding for final mix: 

\begin{verbatim}
MMEncode(M_0...M_(N-1), K_0...K_(N-1), TP, M)
- Let M' = M.
- S_0 = R(16).
- T is the date GMT rounded to the current day.
- LS = Len(M)
- Construct H_0 = MH(V,S_0,T,LS,TP,0,0,0)
- Construct B_0 = Encrypt(H(S_0|''SECRECY_MODE")[0:16],
                          PAD(M,62k-128*N))
- H' = RSA_OAEP_Encrypt(K_0,H_0)|B_0

- For x = 1 to N-1
o S_x = R(16)
o H_x = MH(V, S_x, T, 62kb, 0, 0, 0, M_(x-1))
o H' = RSA_OAEP_Encrypt(K_x, H_x)|
     Encrypt(H(S_x,"SECRECY_MODE")[0:16]|H')
- The final message is (M_(N-1), T, H')
\end{verbatim}

\subsubsection{Single Use Return Blocks}

Encoding a return address to be route to $M_0$ through $M_{N-1}$ to
$M_1$. (The keys are $K_0$ to $K_{N-1}$). The message arriving on $M_0$
should be of type TP.

\begin{verbatim}
MMReplyBlock(M_0...M_(N-1), K_0...K_(N-1), TP)
- MS = R(16)
- GK = H(MS[0:16]|"MASTER_DISTRIBUTION_MODE")
- T is the date GMT that the return block should be valid rounded
  to the day. 
- L = 62*1024 - N*128
- Construct H_0 = MH(V,MS,T,L,Tp,Rt,0,0)
- H' = RSA_OAEP_Encrypt(K_0,H_0)
- for x = 1 to N-1
  o S_x = H(GK|x)[0:16]
  o H_x = MH(V,S_x,T,62*1024,TP,0,0,M_(x-1))
  o H'= RSA_OAEP_Encrypt(K_x, H_x)| 
    Encrypt(H(S_x|"SECRECY_MODE")[0:16],H')
- The return block is (M_(N-1), TP, T, H')
\end{verbatim}

The reply block may be encoded in a variety of ways depending on the
module that uses it, but it should contain all necessary information
to send a message back to the creator: The address of the first mix,
the timestamp for which it is valid, and the reply block itself.

In order to construct a message addressed using the Single Use Return
Block ($M_{N-1}$ , TP, T, H'), a body of length 62*1024 - Len(H') is appended
to the block H' and sent to the node $M_{N-1}$ before the deadline T
is reached. 

\subsubsection{Message processing by each intermediate node}

The standard decoding for an incoming packet is as follows:
If the public key of the mix is $K_x$ and the message received is M.

\begin{verbatim}
MMProcess(K_x, M)
- MH(V,MS,T,S,Tp,Rt,Ru,A) = RSA_OAEP_Decrypt(K_x,M[0:128])
- Check that the OAEP structure is intact.
- Check that protocol version V is supported.
- Check T to make sure the message is valid.
- Calculate US = H(M[0:128]) and check that it has not been seen
  before.
- Store US.
- If Tp = 0
  o Check that S = 62kb
  o D=Decrypt(H(MS|"SECRECY_MODE")[0:16],M[128:62kb-128],0,
              62kb-128)
  o J=Encrypt(H(MS|"JUNK_GENERATION_MODE")[0:16], Z(128))
  o Output message (A,D|J).
- Else
  o Check that A = 0
  o if Rt>0 
    * GK = H(MS[0:16]|"MASTER_DISTRIBUTION_MODE")
    * D = M[128:62kb-128]
    * for x = 1 to Rt-1
    * S_x = H(H(GK|x)[0:16]

    * D = Decrypt(H(S_x,"SECRECY_MODE"), D , x*128, 
                  Len(D)-(x-1)*128)
    * Return (Tp, H(MS,"MODULE_MODE"),D[0:62kb-(N-1)*128]) 
      to application

[Q The few lines above describe how to decode a SURB, addressed to this
mix. Would it be better if the type of message was actually specified
by the real sender rather than the person that constructs the onion?

WARNING: There might be some vulnerabilities in letting the sender
specify the type of message. If any side effect resulting from the
message being processed as a particular type is visible then this is
threatening to the anonymity of the system.]

  o else
    * D=Decrypt(H(MS|"SECRECY_MODE")[0:16],
                       M[128:62kb-128],0,S)[0:S]
    * Return (Tp, H(MS,"MODULE_MODE"),D) to application
\end{verbatim}

\subsubsection{Notes}

If any of the checks and invariants fail then an error message SHOULD
be logged and the packet SHOULD be dropped. The error messages that
occur before the mixing phase MUST be reported as input errors log,
while the errors after the mix stage MUST be reported in the output
errors logs.

For reasons relating to the efficient and stateless engineering of return 
addresses Rijendal in Counter Mode [AppliedC] is used to perform the 
encryption and decryption operations. The encryption and decryption
equations for Counter Mode are:

\begin{verbatim}
Ci = Ek(i + IV) XOR Pi  : To do the encryption.
Pi = Ek(i + IV) XOR Ci  : To do the decryption.

IV = Z(16)
\end{verbatim}

Where Ci is the ith block of cipher text, and Pi is the ith block of
plaintext. Ek is the encryption operation under key k or a block. IV
is an initialization vector set to zero.

The advantage of this mode is that both the encryption and decryption
functions provide secrecy. It is also easy to do random access
decryption that is going to be used to construct stateless return
addresses. Since each key should be different there is no need for an
IV, so we set it to zero. 

\section{Transport issues}

In order to minimize the benefit from eavesdropping on the links 
between mix servers a forward secure transport protocol, MMTP, is used. 
Clients SHOULD implement this protocol if they wish to send anonymous 
messages or receive messages using anonymous reply blocks.

\subsection{Definition of the MixMinion Transfer Protocol}

A special channel should be established between mixes that provides
forward secrecy making it impossible to recognize or decrypt any
message that went through it in the past. In order to establish this
channel one of the two mixes initiates the connection but at the end
of the key exchange protocol the channel is bi-directional.

SSL does not need to be modified in any way to provide forward secure
communications. We will use a mix of ephemeral Diffie Hellman (DH)
keys and key updating to archive our protocol goals.

Outline for SSL based protocol:

\begin{verbatim}
- A invents a new Diffie Hellman key 
  (of at least 1024 bits modulus)
  and makes a certificate for signed by her signing key.
  A then initiates the SSL Handshake protocol with B.
- B invents a DH key and makes a certificate using his signing
  key.
- The SSL handshake protocol proceeds as normal until a session
  key has been established. All communications are then encrypted
  using this session key.

- A sends "INIT",H(M,"INIT") to B.
- B sends "OK", H(H(M,"INIT"),"OK")
- A sends "SEND", M
- B sends "RECEIVED", H(M,"RECEIVED")

- A sends an SSL handshake renegotiation message.
  (and MUST not reuse the same key for 
   transfering another message)
  This updates the session key and overrides the old ones.
\end{verbatim}

\emph{Note:}

The old keys must be permanently overwritten. Special care should be
taken to permanently erase them from the Hard Disk and memory. 

Only secure cipher suites should be accepted by any of the
communicating parties. Diffie Hellman is used for the key exchange
since it allows for quick key generation. The blowfish or 3DES cipher
MAY be used fas the symmetric cipher.

The standard transport mechanism over which the MixMinion Transfer
Protocol is TCP over IP. The standard listening TCP port should be 
number 48099 (until we register a port with www.iana.org)

[Question: Should we request a system (<1023) or user port]

All possible checks should be performed during the transfer protocol
and if any of them fails the connection MUST stop and all state MUST
be deleted. An error MAY be logged. In particular the certificate of
the communication partners should be signed using a key that hashes to
the address hash if it is present in the master header.

\section{Administrative Issues}

\subsection{Node information formats}

Mix nodes need to advertise their address and keys so that nodes can
use them. The format that should be used when describing a mix node
should be the following:

Encoding of Mix Public Keys:

\begin{verbatim}
Content-Type: text/mixminion-ServerKey
Address: [The server address] <LFCR>
Expires: [The date the key expires] <LFCR>
Capabilities: [The capabilities of the mix node] <LFCR>
RSA-E-e: [ I2OSP converted "e" value of the RSA public Encryption 
           key encoded using base64].<LFCR>
RSA-E-n: [ I2OSP converted "n" value of the RSA public Encryption 
           key encoded using base64].<LFCR>
RSA-S-e: [ I2OSP converted "e" value of the RSA public Signature 
           key encoded using base64].<LFCR>
RSA-S-n: [ I2OSP converted "n" value of the RSA public Signature 
           key encoded using base64].<LFCR>
Signature: [The signed SHA-1 of all the previous fields using the 
            mix key in base64]. <LFCR>
\end{verbatim}

The address field can contain more than one addresses separated by
white spaces. The capabilities refer to modules. Clients should not 
send messages relying on capabilities that a mix does not advertise.

The Signature keys MUST NOT be used to encrypt or decrypt material.

We assume that the key exchange happens out of band and we will not
describe a particular protocol to perform it. It is very important
that the key exchange and mix discovery protocols do not break the
security properties of the MixMinion system.

\subsection{Requesting information on a node}

[Q describe standard way that you can request information from a node]

\subsection{Logging format permitted}

Only aggregate per day information should be published. The
log information MUST not be usable to link input with output
messages. A node MAY provide information about the availability and
reliability of other nodes.

[Q If we want to have logs automatically processable we should define
formats here. We need to have some statistics on the mix, such as \#of
messages in and out for the day, failed deliveries for other mixes
during the day, etc ...]

\section{Standard Modules}

All modules have a fixed type number. The contents of the message
along with some shared secret is given to the appropriate module for
processing by the node.

\subsection{DUMP}

[Q specifies that the received message should be silently dropped]

\subsection{INFO}

[Q requests information about the server, or other servers. This can be
transformed into a full in band resource discovery protocols. But do
we really want to have such a thing in this document?]

\subsection{MULTIPART}

[Who knows enough about forward error correction codes (Turbo codes?)
to write a spec about them (Zooko?)? Given the packet size (63kb)
what should be the error correction rate? ]

\section{Security Considerations}

The principal aim of a mix network is
to deliver messages that cannot be linked to a sender or a
recipient. The attacks bellow either try to reduce the unlinkability
provided by the system, or the reliability of the system. Some
attacks concern particular mixes, others concern the mix network as a
whole.

\subsection{Classic mix attacks - [chaum]}

Linkability by Size - If the size of input and output messages can be
used to link particular messages then the mix fails. MixMinion use
fixed length messages and pads the messages after processing.

Linkability by Content - If the content of the message provides any
way of linking the input with the output messages then the mix
fails. This attacks has both a passive variant, but also an active
variant in which an attacker modifies a message to make it
traceable. MixMinion should act as a random function for anyone that
does not know the private key of the server, except the users whose
anonymity is protected. The user that constructs the message headers
knows the secrets used to encrypt and pad the messages and can
therefore link the messages together.

Linkability by order or timing - If the routing strategy for
outputting of the messages provides a way of linking them with
incoming messages then the mix fails. MixMinion does not provide any
guidance but a simple mix using Chaums original proposal in order to
mix the messages: Each mix waits for a batch of valid messages to
arrive, decrypts them and then outputs them in some random
order. Other strategies can be implemented such as pool mixes and
delay mixes.

Linkability by repetition - If a message is processed more than once,
and its decryption and routing is deterministic, then it is obvious to
whom the message is addressed and therefore the message can be
traced. MixMinion implements some duplicates detection
mechanisms. These rely on signature and timestamp to
detect messages that have been processed before, and at the same time
minimize the resources used by to system.

\subsection{N-1 Attack - [BABEL]}

The n-1 attacks can be considered generally as a ``fake cover traffic
attack.'' The opponent tricks the mix to think that it has enough cover
traffic to hide the correspondence of the inputs and output, but has
in fact a way of distinguishing the cover traffic messages form the
real ones. This has as an effect that the anonymity set of the
messages is greatly reduced, and at the limit when only one valid
message is processed the unlinkability of this message is totally
compromised. The standard way of performing this attack is by flooding
the mix, with messages after each valid message it receives. MixMinion
does not provide as a protocol any way of avoiding this
attack. Particular node can introduce cover
traffic to guarantee a minimum size anonymity set (if the output batch
size is always greater than the input, then the opponent cannot know
the exact correspondence of the single input message). The fact the
MixMinion messages expire may also discourage an attacker from trying
this technique if he is ``shy,'' and does not want to be detected.

\subsection{Traffic analysis attacks - Pfitzmann}

Position in MIX route attack - Is a mix is able to know its own
position in the route of a message it can use this information in
order to partition the batches of messages. This attack assumes that
the attacker controls a large number of mixes and is able to trace the
messages that are traveling for a particular number of hops, and select
them as the only potential outputs. MixMinion does not impose a fixed
route to be followed by messages but at the same time mixes should not
be able to know where they are in the path. This makes the attack less
realistic, since we should assume that a majority, or at least a large
number of mixes is honest. A cascade strategy can also be implemented
so that this attack is not possible.

Determining the next MIX when multiple messages are sent using the
same path - If two messages are sent along the same path an attacker
can use the intersection of their anonymity sets in order to find the
route they traveled and the final recipient. Using the same path is
discouraged, unless a safe strategy such as cascade mixing is
implemented. 

Message blocking - If an attacker knows that a message is going to be
sent to him, as part of a larger data stream, and wants to detect the
sender, he may block messages until some expected messages are
missing. This is more of a ``traffic confirmation'' attack, and
MixMinion does not attempt to address it.

\subsection{Traffic Analysis attacks - Freedom IH2001}

Some of the attacks presented in [IH2001 - Freedom] concern
Onion-routing type systems, and do not directly affect MixMinion. The
packet counting attack only affects onion routing services with
first-come first-served routing policies. The attacks on traffic
shaping are very close to the n-1 attacks an underline the need for a
shaping strategy that always has some dummy traffic on the
links. MixMinion is not concerned with this and should work with or
without such strategies (although it would be weaker if it does not
implement them). The Latency attacks should not work very well if a
tight expiry date is set on the packets since the stream is simply
going to be dropped, which would probably terminate any persistent
connections. This is a traffic confirmation attack that cannot be
completely avoided. The clogging attacks are also a ``traffic
confirmation attack'' by which one sends traffic along the suspected
route of a stream to see the results in its latency. If the stream
under observation has its latency characteristics changed, then it is
inferred that the guess was correct.

\subsection{Traffic confirmation attacks}

As described in many cases above traffic confirmation attacks are
performed when there is already a suspicion that a particular user is
associated with some communication. These attacks are very difficult
to defend against. The most crude way of performing these attacks is
to disconnect the the suspect to see if the communication stream stops.

\subsection{Reputation and path selection attacks}

The current MixMinion network and some recent proposals rely on the
fact that statistics of reliability are compiled for mixes, so that
reliable routes are selected. If Denial of Service attacks can
influence these reputation systems, it may be the case that an
attacker could influence its victim to choose a compromised route
through the network.

\subsection{Legal attacks}

GAK legislation can be used to compel mixes to decrypt traffic or give
their keys. Logs can be seized, but should do not contain any
information that would compromise the anonymity of users.

\section{Issues not addressed}

Key distribution, revocation infrastructure

\section{Conclusions}
\section{References}
\end{document}