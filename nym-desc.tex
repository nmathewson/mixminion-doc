Specification of a nym server running on top of Mixminion.

We assume that the underlying anonymizing network (in this case
MixMinion) provides the following anonymous communications
primitives:

(A) -> B: M   - An anonymous party A sends a message to a named party B
                the message M. M has a maximum length of L.
A -> (B): M   - An named party A sends a message M to an anonymous
                recipient B. M has a maximum length of L.
(A) -> (B): M - An anonymous sender A sends a message M to an
                anonymous recipient B. M has a maximum length of L.
(B)[ID]       - A Single Use Reply Block (SURB) destined to B. A label
                ID can be used to differentiate between different
                SURBs.

Implementing the anonymous primitives using MixMinion

[...]

The requirements of the Nym Server are:

- The Nym server allows users without any special software (besides
  standard SMTP) to send email [RFC2821] to anonymous recipients.
- The Nym server should not be trusted or relied upon to maintain the
  anonymity of the users. Although it should be trusted not to deny
  service to honest users.
- The Nym server should not be required to maintain more state than
  its administrator wishes. In particular the administrator should
  have the option not to store messages in clear or SURBs.
- The nym server should provide reliable service, even if the
  anonymous communication channels used are not. In particular it
  should ensure that if users follows the protocols correctly none of
  their mail will be dropped.
- Clients that do not benefit from any anonymity properties do not
  need any special software.
- Neither clients or nym servers need to actually run anonymous
  remailer nodes themselves. They can function by simply acting as
  clients to the anonymous network, sending and receiving anonymous
  messages. 

The Nym model

Users run special software, the Nym client (that understands MixMinion
and Nym server commands), and anonymously create a fresh nym account on
the nym server. After this setup phase any email received by the Nym
server for the nym is anonymously forwarded to the appropriate
user. The Nym server makes sure that the message has correctly arrived
to the user before it is deleted.

  Anonymous                      Nym             Rest 
     User    <---MixMinion---> Server <--SMTP-- of the
 (Nym Client)                                    World

Client and Server Databases

The objective of the design is to minimize the information required to
be stored by both the client and server, while maintaining high
reliability. 

Client:

list of {
  Full Nym Name (Nym)
  Server Address (A)
  Server Verification key (KSv)
  Client Signature Key (KCs) + Verification key (KCv)
  Client Decryption Key (KCd) + Encryption key (KCe)
  Kill phrase (KP)
}

Server:

Server Signature Key (KSs) + Verification key (KSv)

list of {
  Full Nym Name (Nym)
  Client Verification key (KCv)
  Client Encryption key (KCe)
  Hash of Kill Phrase (H(KP))
  Should it keep SURBS? (KS?)
  list of i = 1..n {
    SURBS ((Nym)[i])
  }
  list of j=1..m {
    Message (M[j])
  }
}

Nym Server Public Information

A Nym server needs to advertise its address for both SMTP and
anonymous access, and a public verification key to verify messages
sent by it. Additionally policy files need to be available concerning
the creation of nyms (acceptable names, if it stores SURBs or not),
and the conditions of service (max. messages sizes, how often SURBs
should be given, lifetime of accounts, etc.) 

Protocols

Initialization:

The client chooses a Nym Name (Nym) which will be the base of its
email address on the nym server. The client also provides the server
with public keys to encrypt (KCe) and verify (KCv) messages. The Kill
Phrase is used in case the Signature keys are stolen or lost for the
user to be able to destroy the Nym account. 

In order to create the account the Client provides some means for the
server to contact them back, in the form of SURB's.

(C) -> NymServer: {CREATE,Nym, KCv, A, KSv, KCe, H(KP), KS?,n,(C)[1]..(C)[n]}KCs

The Nym Server can deny an account creation request for a variety of
reasons, which should be communicated to the used using the
``message''. The server could be full, the Nym already in use, or rude.
It is important for Nym server operators to have a clear policy on
which nyms are acceptable and which are not.

NymServer -> (C)[x]: {ACCEPT|DENY, A, KSv, Nym, KCv, Message}KSs

On Server receiving messages:

The Nym Server runs an SMTP [RFC2821] service. When a message
received is addressed to a particular nym, it is encrypted under the
nym's key (KCe) and stored on permanent storage. Depending on whether
the Server possesses any SURBs corresponding to the nym it can forward
the messages or simply wait for more SURBs. At the end of any MESSAGE
the nym server can specify the number of additional SURBs that it
requires to flush all messages. The client is not obliged to send the
specified number of SURBs, but the server may stop receiving mail for
this particular Nym, or drop the account.

A -> NymServer: M (with SMPT to: NymB)
NymServer -> (B): {MESSAGE, A, KSv, NymB, KCv,
                   n,{M}KCe[1]..{M}KCe[n],m, info}KSs
(B) -> NymServer: {RECEIVED,  NymB, KCv, A, KSv, n,
                   H({M}KCe[1])..H({M}KCe[n]), m, (B)[1]..(B)[m]}

To kill a Nym:

(A) -> NymServer: KILL,NymA, KP

Since a nym's signature key might be compromised or lost it is
important to have a backup but abuse proof mechanism to destroy nyms
on the server, along with all the stored information. Therefore the
policy should be that if the server receives a kill messages with a
``Kill phrase'' with a hash that matches the one stored for the nym,
it must destroy all information about the concerned nym.

Sending messages

[Since we have decided that the SMTP mechanism is general we do not
  need a specific sending ability for the Nym server]

Multiplexed reliable mix communications and Information dispersal techniques

[Since many different protocols might be using the anonymous
  communication infrastructure it could be helpful if we create a port
  abstraction using either an identifier at the beginning of the
  payload or different mailbox names for the Local delivery present in
  the last header of the message] 
[In order to avoid retransmissions, that could help traffic analysis,
  an idea would be to send the messages of the above protocol using a
  forward error correcting scheme. The Server, and clients, could
  measure the reliability of the network, by sending messages to
  themselves, and encode each message in a (n out of m) fashion. Maybe
  someone who is knowledgeable about such schemes would like to comment?]

References

[RFC2821] J. Klensin, Editor, Simple Mail Transfer Protocol

