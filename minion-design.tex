
\documentclass{llncs}

\newcommand{\workingnote}[1]{}        % The version that hides the note.
%\newcommand{\workingnote}[1]{(**#1)}   % The version that makes the note visible.

%\newif\ifpdf
%\ifx\pdfoutput\undefined
%   \pdffalse
%\else
%   \pdfoutput=1
%   \pdftrue
%\fi

\begin{document}
%\setstretch{2.0}

%% Use dvipdfm instead. --DH
%\ifpdf
%  \pdfcompresslevel=9
%  \pdfpagewidth=\the\paperwidth
%  \pdfpageheight=\the\paperheight
%\fi

\title{The Mixminion Anonymous Remailer}
\author{George Danezis\inst{1} \and Roger Dingledine\inst{2} \and Nick Mathewson\inst{2}}
\institute{Cambridge
\email{(george.danezis@cambridge)}
\and
The Free Haven Project
\email{(arma@mit.edu)}
% add some more people here
}
\maketitle
\pagestyle{empty} 
  
\begin{abstract}

We describe a packet-based anonymous remailer protocol which supports
single-use reply blocks and includes link-level encryption to provide
forward anonymity. We include justification for various design decisions
and a detailed description of attacks and defenses. And some other stuff.

\end{abstract}

Keywords: anonymity, peer-to-peer, remailer, store-and-forward, reply block %, ...

%%%%%%%%%%%%%%%%%%%%%%%%%%%%%%%%%%%%%%%%%%%%%%%%%%%%%%%%%%%%%%%%%%%%%%%

\section{Introduction}
\label{sec:intro}

Chaum first introduced anonymous remailer designs over 20 years ago
\cite{chaum-mix}. The research community has since introduced many
new designs and proofs, and discovered a variety of new attacks, but
the state of deployed remailers has changed remarkably little since
Cottrell published his Mixmaster software \cite{mixmaster-attacks} eight years
ago. Part of that is due to the liability involved in running a remailer
node on the Internet, and part is due to the complexity of the current
infrastructure --- it is very hard to add new experimental features
to the current software.

The Mixminion project aims to deploy a cleaner updated remailer design
in the same spirit as Mixmaster, with the goals of expanding deployment
and providing a research base for experimental features. Specifically, we:

\begin{itemize}
\item Introduce a new primitive called a \emph{single-use reply block}
(SURB), and describe how to build higher-level systems such as nymservers
using these SURBs. Mixmaster provides no support for replies, instead
relying on the older and less secure cypherpunk remailer design
\cite{cypherpunk-remailer}.  By integrating reply capabilities into
Mixminion, we can finally retire the cypherpunk type 1 remailer network.
\item Introduce link-level encryption with ephemeral keys to ensure
forward anonymity for each message.
\item Provide flexible delivery schemes --- rather than just allowing
delivery to mail or usenet, we allow designers to add arbitrary modules to
handle incoming messages. By separating the core mixing architecture from
these higher-level modules, we can limit their influence on the anonymity
properties of the system.
\item Describe a \emph{reputation server} design to give users more
information about the current state and reliability of Mixminion servers.
% \item probably some more
\end{itemize}

Many of our design decisions impacted anonymity in surprising ways. Herein
we document and analyze some of these influences to provide more intuition
to developers and users.

% ...

%%%%%%%%%%%%%%%%%%%%%%%%%%%%%%%%%%%%%%%%%%%%%%%%%%%%%%%%%%%%%%%%%%%%%%%

\section{Goals and Assumptions}

% and non-goals
% threat models, etc

http://archives.seul.org//mixminion/dev/Mar-2002/msg00004.html
http://archives.seul.org//mixminion/dev/Mar-2002/msg00014.html

%%%%%%%%%%%%%%%%%%%%%%%%%%%%%%%%%%%%%%%%%%%%%%%%%%%%%%%%%%%%%%%%%%%%%%%

\section{Related Work}

Just a matter of writing it up. Mixes. Mixmaster/Babel. Flash/StopandGo.

I'll get to it later on if others don't. -RRD

%%%%%%%%%%%%%%%%%%%%%%%%%%%%%%%%%%%%%%%%%%%%%%%%%%%%%%%%%%%%%%%%%%%%%%%

\section{Design Overview}

\subsection{Packet structure, how packets travel}

This subsection probably should wait until we've got a better plan for
how to do replies.

\subsection{Why forward and reply messages are secure}

and same with this one.

\subsection{Link-level encryption and what it gets us}

Diffie-Hellman with ephemeral keys. OpenSSL. Some discussion of how
this makes purely passive adversaries worse off, but really not that
much worse off because they can still watch the number of characters
going by on the channel.

\subsection{Message types and delivery modules}

This one is decided as well. Just needs to be written up.

One issue: delivery information in header or in payload?
There are tradeoffs to each. Describe both and we'll pick one down the road.

\subsection{Exit policies and abuse}

Looks quite straightforward. More generally, this should be a discussion
about capabilities for each mix.

How do clients communicate with a mix to learn its capabilities? Or does
the mix communicate capabilities to the reputation server and the client
gets them from there? Or both. Ties in with reputation server section
below.

%%%%%%%%%%%%%%%%%%%%%%%%%%%%%%%%%%%%%%%%%%%%%%%%%%%%%%%%%%%%%%%%%%%%%%%

\section{Reputation Servers}

initially the reputation servers are just to track participating
mixes
and then hand out keys for them
and replace the pinging servers
and then do pinging themselves

reliability for paths, nodes, reply blocks

http://archives.seul.org/mixminion/dev/Apr-2002/msg00002.html

%%%%%%%%%%%%%%%%%%%%%%%%%%%%%%%%%%%%%%%%%%%%%%%%%%%%%%%%%%%%%%%%%%%%%%%

\section{Nym management and single-use reply blocks}

we've got two competing notions for how to do nymservers, and i think
it's becoming clear that one is better than the other. good to describe
them both.
http://archives.seul.org/mixminion/dev/Apr-2002/msg00047.html
and following thread

\subsection{Nymservers}

\subsection{Long-term nyms: how to choose paths for reply blocks}

This question is hard. We're going to have to argue about it for a
while more, I think.

%%%%%%%%%%%%%%%%%%%%%%%%%%%%%%%%%%%%%%%%%%%%%%%%%%%%%%%%%%%%%%%%%%%%%%%

\section{Maintaining anonymity sets}

\subsection{Transmitting large files with Mixminion}

http://archives.seul.org/mixminion/dev/Apr-2002/msg00031.html

\subsection{key rotation and anonymity}

http://archives.seul.org/mixminion/dev/Apr-2002/msg00047.html

%%%%%%%%%%%%%%%%%%%%%%%%%%%%%%%%%%%%%%%%%%%%%%%%%%%%%%%%%%%%%%%%%%%%%%%

\section{Implementation choices}

some details about how to build it. logging and statistics? etc.

nick?

%%%%%%%%%%%%%%%%%%%%%%%%%%%%%%%%%%%%%%%%%%%%%%%%%%%%%%%%%%%%%%%%%%%%%%%

\section{Attacks and Defenses}

my aim here is to do something akin to pages 13-15 of
http://freehaven.net/doc/casc-rep/casc-rep.ps

%%%%%%%%%%%%%%%%%%%%%%%%%%%%%%%%%%%%%%%%%%%%%%%%%%%%%%%%%%%%%%%%%%%%%%%

\section{Future Directions}

%%%%%%%%%%%%%%%%%%%%%%%%%%%%%%%%%%%%%%%%%%%%%%%%%%%%%%%%%%%%%%%%%%%%%%%

\section*{Acknowledgements}

%%%%%%%%%%%%%%%%%%%%%%%%%%%%%%%%%%%%%%%%%%%%%%%%%%%%%%%%%%%%%%%%%%%%%%%

\bibliographystyle{plain} \bibliography{minion-design}

\end{document}

