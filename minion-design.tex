
\documentclass{llncs}

\newcommand{\workingnote}[1]{}        % The version that hides the note.
%\newcommand{\workingnote}[1]{(**#1)}   % The version that makes the note visible.

%\newif\ifpdf
%\ifx\pdfoutput\undefined
%   \pdffalse
%\else
%   \pdfoutput=1
%   \pdftrue
%\fi

\begin{document}
%\setstretch{2.0}

%% Use dvipdfm instead. --DH
%\ifpdf
%  \pdfcompresslevel=9
%  \pdfpagewidth=\the\paperwidth
%  \pdfpageheight=\the\paperheight
%\fi

\title{The Mixminion Anonymous Remailer}
\author{George Danezis\inst{1} \and Roger Dingledine\inst{2} \and Nick Mathewson\inst{2}}
\institute{Cambridge
\email{(george.danezis@cambridge)}
\and
The Free Haven Project
\email{(arma@mit.edu)}
% add some more people here
}
\maketitle
\pagestyle{empty} 
  
\begin{abstract}

We describe a packet-based anonymous remailer protocol which supports
single-use reply blocks and includes link-level encryption to provide
forward anonymity. We include justification for various design decisions
and a detailed description of attacks and defenses. And some other stuff.

\end{abstract}

Keywords: anonymity, peer-to-peer, remailer, store-and-forward, reply block %, ...

%%%%%%%%%%%%%%%%%%%%%%%%%%%%%%%%%%%%%%%%%%%%%%%%%%%%%%%%%%%%%%%%%%%%%%%

\section{Introduction}
\label{sec:intro}

Chaum first introduced anonymous remailer designs over 20 years ago
\cite{chaum-mix}. The research community has since introduced many new
designs and proofs, and discovered a variety of new attacks, but the
state of deployed remailers has changed remarkably little since Cottrell
published his Mixmaster software \cite{mixmaster-attacks} eight years
ago. Part of the difficulty in expanding the deployed remailer base is
due to the liability involved in running a remailer node on the Internet,
and part is due to the complexity of the current infrastructure ---
it is very hard to add new experimental features to the current software.

The Mixminion project aims to deploy a cleaner updated remailer design
in the same spirit as Mixmaster, with the goals of expanding deployment,
documenting our design decisions and how well they stand up to all known
attacks, and providing a research base for experimental features. We
describe our overall design in Section \ref{sec:design}, including two
designs for a new primitive called a \emph{single-use reply block}
(SURB).  Mixmaster provides no support for replies, instead relying
on the older and less secure cypherpunk type I remailer design
\cite{cypherpunk-remailer}. By integrating reply capabilities into
Mixminion, we can finally retire the type I remailer network.

We go on in Section \ref{sec:rep-servers} to describe a design for
Reputation Servers to track and distribute remailer availability,
performance, and key information, and then describe in Section
\ref{sec:nymservers} how to build higher-level systems such as nymservers
using SURBs. We introduce link-level encryption with ephemeral keys to
ensure forward anonymity for each message. We also provide flexible
delivery schemes --- rather than just allowing delivery to mail or
usenet, we allow designers to add arbitrary modules to handle incoming
messages. By separating the core mixing architecture from these
higher-level modules, we can limit their influence on the anonymity
properties of the system.

Mixminion aims to be a best-of-breed remailer which uses conversative
design approaches to provide security against most known attacks.
Many of our design decisions impacted anonymity in surprising ways. Herein
we document and analyze some of these influences to provide more intuition
to developers and users.

%%%%%%%%%%%%%%%%%%%%%%%%%%%%%%%%%%%%%%%%%%%%%%%%%%%%%%%%%%%%%%%%%%%%%%%

\section{Related Work}

\subsection{MIX-nets}

Chaum introduced the concept of a MIX-net for anonymous communications
\cite{chaum-mix}. A MIX-net consists of a group of servers, called MIXes
(or MIX nodes), each of which is associated with a public key. Each
MIX receives encrypted messages, which are then decrypted, batched,
reordered, stripped of the sender's name and identifying information, and
forwarded on. Chaum also proved security of MIXes against a \emph{passive
adversary} who can eavesdrop on all communications between MIXes but is
unable to observe the reordering inside each MIX.

Current research directions on MIX-nets include ``stop-and-go'' MIX-nets
\cite{kesdogan}, distributed ``flash MIXes'' \cite{flash-mix} and their
weaknesses \cite{desmedt,mitkuro}, and hybrid MIXes \cite{hybrid-mix}.

\subsection{Deployed Remailer Systems}

The first widespread public implementations of MIXes were produced by the
cypherpunks mailing list. These ``Type I'' \emph{anonymous remailers}
were inspired both by the problems surrounding the {\tt anon.penet.fi}
service \cite{helsingius}, and by theoretical work on MIXes. Hughes wrote
the first cypherpunks anonymous remailer \cite{remailer-history}; Finney
followed closely with a collection of scripts which used Phil Zimmermann's
PGP to encrypt and decrypt remailed messages. Later, Cottrell implemented
the Mixmaster system \cite{mixmaster}, or ``Type II'' remailers, which
added message padding, message pools, and other MIX features lacking
in the cypherpunk remailers. At about the same time, Gulcu and Tsudik
introduced the Babel system \cite{babel}, which also created a practical
remailer design (although one that never saw widespread use).

%%%%%%%%%%%%%%%%%%%%%%%%%%%%%%%%%%%%%%%%%%%%%%%%%%%%%%%%%%%%%%%%%%%%%%%

%\section{Goals and Assumptions}
% covered below, at least in part.

% and non-goals
% threat models, etc

%http://archives.seul.org//mixminion/dev/Mar-2002/msg00004.html
%http://archives.seul.org//mixminion/dev/Mar-2002/msg00014.html

%%%%%%%%%%%%%%%%%%%%%%%%%%%%%%%%%%%%%%%%%%%%%%%%%%%%%%%%%%%%%%%%%%%%%%%

\section{Design Overview}
\label{sec:design}

Mixminion aims to bring together the current best approaches for providing
anonymity in a batching message-based MIX environment. We don't aim
to provide low-latency connection-oriented services like Freedom
\cite{freedom} or Onion Routing \cite{onion-routing} --- while those
designs are more effective for common activities like anonymous web
browsing, it seems they cannot provide the level of strong anonymity of
slower message-based services [Do we, uh, want to cite that?]. Indeed, we
further intentionally restrict the set of options for users: we provide
only one cipher, and we avoid extensions that would help an adversary
divide the anonymity set.

We chose to drop backward-compatibility with Mixmaster and the cypherpunk
remailer systems, in order to provide a simple extensible design. At
the same time, we provide a new feature: a reply block mechanism which
is as secure as forward messages.

Reusable reply blocks are security risks --- by their very nature they
let people send multiple messages to them. These multiple messages can be
used to very quickly trace the recipient's path: if two incoming batches
both include a message to the same reply block, then the next hop must
be in the intersection of both outgoing batches.

%Mixmaster solves the problem by not providing reply functionality,
%and instead falling back on the less secure multiple-use reply blocks
%provided by the type I remailer network.

The rest of this section describes the header structure and the
protocol for building and processing messages. In particular, we
focus on two competing ways of providing secure reply functionality,
and the tradeoffs and design decisions surrounding each approach. In
the first approach, we provide a mechanism for doing replies where a
reply message is indistinguishable from a forward message. Because this
approach introduces some attacks which we cannot adequately address, we
then propose a second approach which allows forward and reply messages
to be distinguished but provides better overall security.


\subsection{Approach one: the `header swap' method}

Making forwards and replies indistinguishable prevents an adversary from
dividing the message anonymity sets into two classes. In particular, if
there are very few replies during a given period relative to the total
number of messages, an adversary controlling some of the MIXes in the
network can more easily trace the path of each reply --- even though
the batches may be large, the number of replies in each batch will be
quite small.

At the same time, protocols like Mixmaster include hashes of the entire
message in each header. Each hop in the path checks the integrity of
the header and payload, and drops the message immediately if it's been
altered. But since the author of the reply block is not the one writing
the payload, these hashes can no longer be used. Indeed, since we choose
to make forward message and replies indistinguishable, we cannot provide
hashes for forward messages either. This choice introduces a new class
of attacks known as \emph{tagging attacks}.






\subsection{Approach two: the `distinguish replies' method}

%David? :)

\subsection{Link-level encryption and what it gets us}

Diffie-Hellman with ephemeral keys. OpenSSL. Some discussion of how
this makes purely passive adversaries worse off, but really not that
much worse off because they can still watch the number of characters
going by on the channel.

\subsection{Message types and delivery modules}

This one is decided as well. Just needs to be written up.

One issue: delivery information in header or in payload?
There are tradeoffs to each. Describe both and we'll pick one down the road.

\subsection{Exit policies and abuse}

Looks quite straightforward. More generally, this should be a discussion
about capabilities for each mix.

How do clients communicate with a mix to learn its capabilities? Or does
the mix communicate capabilities to the reputation server and the client
gets them from there? Or both. Ties in with reputation server section
below.

%%%%%%%%%%%%%%%%%%%%%%%%%%%%%%%%%%%%%%%%%%%%%%%%%%%%%%%%%%%%%%%%%%%%%%%

\section{Reputation Servers}
\label{sec:rep-servers}

initially the reputation servers are just to track participating
mixes
and then hand out keys for them
and replace the pinging servers
and then do pinging themselves

reliability for paths, nodes, reply blocks

http://archives.seul.org/mixminion/dev/Apr-2002/msg00002.html

%%%%%%%%%%%%%%%%%%%%%%%%%%%%%%%%%%%%%%%%%%%%%%%%%%%%%%%%%%%%%%%%%%%%%%%

\section{Nym management and single-use reply blocks}
\label{sec:nymservers}

we've got two competing notions for how to do nymservers, and i think
it's becoming clear that one is better than the other. good to describe
them both.
http://archives.seul.org/mixminion/dev/Apr-2002/msg00047.html
and following thread

\subsection{Nymservers}

\subsection{Long-term nyms: how to choose paths for reply blocks}

This question is hard. We're going to have to argue about it for a
while more, I think.

%%%%%%%%%%%%%%%%%%%%%%%%%%%%%%%%%%%%%%%%%%%%%%%%%%%%%%%%%%%%%%%%%%%%%%%

\section{Maintaining anonymity sets}

\subsection{Transmitting large files with Mixminion}

http://archives.seul.org/mixminion/dev/Apr-2002/msg00031.html

\subsection{key rotation and anonymity}

http://archives.seul.org/mixminion/dev/Apr-2002/msg00047.html

%%%%%%%%%%%%%%%%%%%%%%%%%%%%%%%%%%%%%%%%%%%%%%%%%%%%%%%%%%%%%%%%%%%%%%%

\section{Implementation choices}
\label{sec:implementation}

some details about how to build it. logging and statistics? etc.

nick?

%%%%%%%%%%%%%%%%%%%%%%%%%%%%%%%%%%%%%%%%%%%%%%%%%%%%%%%%%%%%%%%%%%%%%%%

\section{Attacks and Defenses}
\label{sec:attacks}

my aim here is to do something akin to pages 13-15 of
http://freehaven.net/doc/casc-rep/casc-rep.ps

%%%%%%%%%%%%%%%%%%%%%%%%%%%%%%%%%%%%%%%%%%%%%%%%%%%%%%%%%%%%%%%%%%%%%%%

\section{Future Directions}
\label{sec:conclusion}

%%%%%%%%%%%%%%%%%%%%%%%%%%%%%%%%%%%%%%%%%%%%%%%%%%%%%%%%%%%%%%%%%%%%%%%

\section*{Acknowledgements}

%%%%%%%%%%%%%%%%%%%%%%%%%%%%%%%%%%%%%%%%%%%%%%%%%%%%%%%%%%%%%%%%%%%%%%%

\bibliographystyle{plain} \bibliography{minion-design}

\end{document}

